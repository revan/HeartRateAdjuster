    Since Android provides full support for SQLite databases, it is the type of storage that we have chosen for the application. The wide variety of fully-developed features allows us to focus more on the actual organization and management of the data in relation to the other modules. All that is needed is a simple call to the data base to retrieve the raw data, and the custom designed objects illustrated in the Class Diagram then do their own processing on the data. SQLite allows us to store all the data specific to application on the device itself, which is advantageous for a mobile application such as ours. The goal is for the user to be able to record and view his workout data without having to use any external devices other than his phone and the chest strap, and internal data storage via the SQLite database allows our application this benefit. \\
        The database will be accessible only to the Data Manager and the Data Assembler. In regards to the Data Manager, the only interactions with the database will be to store the initial state of the system, store the current music track, and store the current heart rate. It will not retrieve anything from the database, because that is the purpose of the Data Assembler. The Data Assembler is the other object that will interact with the database. It will issue requests for the various data that the UI would like to graph, which include the heart rate, the current times, and the songs. Thus, the Data Manager and Data Assembler are the only objects that have direct access to the database.

